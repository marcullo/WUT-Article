\section{Wprowadzenie}
\label{sec:intro}

Szablon Lecture Notes in Computer Science (LNCS) można wykorzystać do składania i edycji krótkich form tekstowych \cite{ref:lncs}. Dla potrzeb realizacji założeń Seminarium Dyplomowego, przygotowano bliźniacze, zmodyfikowane wersje z~wykorzystaniem języka \LaTeX~\cite{ref:latex} i~dokumentu Microsoft Word (z~makrami), które powinny pozytywnie wpłynąć na komfort pracy Studenta w realizacji zadania.

\subsection{Istota zadania}
\label{subsec:essence}

Jednym z wymagań przedmiotu Seminarium Dyplomowe jest opracowanie artykułu naukowego zawierającego treść referatu dotyczącego Pracy Dyplomowej Studenta. Dopuszcza się przygotowanie tekstu na temat informatyki, niezwiązany z~pracą dyplomową. W tym przypadku należy dowieść istotnego wkładu Studenta (np. poprzez analizę porównawczą metod lub autorskie eksperymenty).

Istotą zadania jest opracowanie nowego artykułu, dotyczącego nieprezentowanych dotychczas zagadnień. Na ocenę dokumentu wpływa jakość opracowania, w tym:

\begin{itemize}[noitemsep]
	\item spełnienie wymagań formalnych,
	\item stosowalność proponowanych rozwiązań,
	\item sposób użycia języka naukowo-technicznego,
	\item brak lub obecność błędów ortograficznych/literowych,
	\item poprawność formatowania tekstu.
\end{itemize}

\begin{note}
	Artykuł z rażącymi błędami redakcyjnymi lub językowymi, znamionującymi brak elementarnej staranności autora, \textbf{może być oceniony negatywnie bez pełnej weryfikacji tekstu}.
\end{note}

\begin{note}
	Treści w prezentacji, w szczególności ilustracje i~schematy blokowe, powinny być oryginalne lub umieszczone w~sposób nienaruszający prawa cytatu.
\end{note}

\subsection{Przygotowanie artykułu}
\label{subsec:preparation}

Student jest zobowiązany do przygotowania i~dostarczenia dokumentu PDF, sformatowanego zgodnie z~szablonem, umożliwiającego nanoszenie komentarzy, stosując zasady opisane poniżej.

\begin{table}
	\vspace{-4mm}
	\caption{
		Zasady przygotowania artykułu.
	}
	\begin{center}
		\begin{tabular}{AB}
			\hline
			Wymagania formalne & 6-8 stron tekstu przy nominalnym formatowaniu. Format papieru A4, marginesy pionowe i poziome, krój i~wielkość czcionek, interlinia, odstępy i~wcięcia akapitów, numeracja sekcji i~stron, wyrównanie, dzielenie wyrazów zdefiniowane przez szablon.\\
			\hline

			Tytuł & Możliwie krótki, zrozumiały i~adekwatny do zawartości artykułu.\\
			\hline

			Streszczenie & Powinno określać cel, zakres i~strukturę artykułu, a~co najmniej anonsować jego treść. Powinno zawierać słowa kluczowe, charakteryzujące przedmiot artykułu.\\
			\hline

			Treść & Zorganizowana w sekcje o czytelnej strukturze logicznej, zawartość sekcji adekwatna do ich tytułów. Powinna wskazywać źródła cytowanych stwierdzeń, hipotez, prognoz, ilustracji, tabel.\\
			\hline

			Podsumowanie & Adekwatne do celu i zakresu artykułu. Powinno zawierać wnioski, a co najmniej syntetyczną rekapitulację treści.\\
			\hline

			Bibliografia & Powinna wskazywać cytowane źródła w~sposób umożliwiający czytelnikowi dotarcie do ich treści. W przypadku źródeł internetowych powinna wskazywać datę odsłony.\\
			\hline
		\end{tabular}
	\end{center}
	\label{tab:rules}
	\vspace{-6mm}
\end{table}

\noindent Należy zadbać o czytelność:

\begin{itemize}[nosep]
	\item wskazywać rozwinięcia używanych skrótów, wyjaśniać specjalistyczne pojęcia przy pierwszym użyciu,
	\item unikać zbyt długich zdań, wtrąceń, przypisów dolnych i innych konstrukcji utrudniających czytanie,
	\item unikać określeń potocznych i żargonowych,
	\item zwrócić uwagę na łamanie linii i stron.
\end{itemize}
