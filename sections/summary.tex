\section{Podsumowanie}
\label{sec:summary}

Jednym z~elementów pracy twórczej Studenta realizującego przedmiot Seminarium Dyplomowe jest dostarczenie nowego artykułu, dotyczącego nieprezentowanych dotychczas zagadnień, przygotowanego zgodnie z szablonem, w formacie PDF, umożliwiającego nanoszenie komentarzy. Autor powinien odwołać się do wszystkich pozycji literatury z~poszanowaniem prawa cytatu. Na ocenę wpływa jakość opracowania i~czytelność. Dokument z~rażącymi błędami redakcyjnymi lub językowymi może być oceniony negatywnie bez pełnej weryfikacji tekstu.

Pierwszy akapit sekcji powinien wystąpić z~pominięciem wcięcia, kolejne - powinny być wcięte. Preferuje się korzystanie z~dwóch lub trzech poziomów zagnieżdżeń. Należy zwrócić uwagę na poprawne łamanie linii i~stron, między innymi przejście do nowej strony na granicy akapitów i~unikanie odwołań do ilustracji/tabel na kolejnych stronach.

Zalecanymi uzupełnieniami treści są tabele, rysunki, schematy, wykresy, wnioskowania logiczne, pseudokod i~cytowania. Rozmiar wspomnianych obiektów powinien być adekwatny do otaczającej treści, a~użycie uzasadnione. Zbyt krótkie nawiązania w~treści do nich mogą być przejawem niedokładności autora, a~zbyt długie - utrudniać zrozumienie jego rozważań.

Należy oszczędnie i~precyzyjnie dowodzić i~wnioskować, aby uporządkować tok rozumowania. Artykuł powinien mieć wyraźną strukturę logiczną, zwykle w~postaci rozumowania dedukcyjnego (\emph{od ogółu do szczegółu}). Po krótkim wprowadzeniu prezentowane są np. cechy konkretnego rozwiązania z~wykorzystaniem adekwatnej metodologii (np. porównanie rozwiązań przy użyciu kryterium ilościowego). Następuje detaliczna analiza tematu, po czym Czytelnik zobowiązany jest do syntetycznej rekapitulacji treści.

Więcej informacji na temat dobrych praktyk tworzenia artykułu zainteresowany Czytelnik może znaleźć w~regulaminie przedmiotu i~poradni językowej PWN~\cite{ref:pwn}. Niniejszy materiał przeznaczony jest dla Studentów realizujących przedmiot Seminarium Dyplomowe w~Instytucie Informatyki Politechniki Warszawskiej. W~przypadku chęci zgłoszenia usterki w~pierwszej kolejności sugeruje się kontakt z~prowadzącym, następnie zamieszczenie komentarza i~propozycji zmiany w~repozytorium projektu \cite{ref:github}.
