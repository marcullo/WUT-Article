\section{Podsumowanie}
\label{sec:summary}

Najmniej wartościowym, dopuszczalnym sposobem podsumowania jest syntetyczna rekapitulacja treści. Wartością dodaną, świadczącą o~jakości puenty jest interpretacja uzyskanych wyników, w~szczególności odniesienie ich do rezultatów innych prac i~wyjaśnienie ewentualnych różnic. Podsumowanie jest właściwym miejscem do przedstawienia rezultatów w~szerszym kontekście (np.~biznesowym lub społecznym) i~wskazania kierunków dalszych prac. Poniżej podsumowano (pokrótce) artykuł z~uwzględnieniem powyższych stwierdzeń.

Jednym z~elementów pracy twórczej studenta realizującego Seminarium Dyplomowe~2 jest dostarczenie nowego artykułu, dotyczącego nieprezentowanych dotychczas (na seminarium) zagadnień, przygotowanego zgodnie z~szablonem. Na ocenę wpływa jakość opracowania i~czytelność. Należy zwrócić uwagę na poprawność formatowania, między innymi właściwe łamanie linii i~stron oraz unikanie odwołań do ilustracji/tabel na kolejnych stronach. Artykuł powinien mieć wyraźną strukturę logiczną, zakończoną syntetyczną informacją o~zakresie zrealizowanych prac i~wartością dodaną, opisaną w~poprzednim akapicie.

Dokument opracowano w~odpowiedzi na zapotrzebowanie na jednolity szablon artykułu w~języku \LaTeX~i~programie Microsoft Word zgłoszone przez grupę studentów realizujących przedmiot, którego niniejszy tekst dotyczy. Do cech wyróżniających artykułu można zaliczyć:

\begin{itemize}
	\item nowatorski opis zasad składania dokumentu z~wykorzystaniem szablonu LNCS w~Instytucie Informatyki PW (wedle stanu wiedzy autora),
	\item zawarcie istoty zadania i~zasad przygotowania artykułu w~logicznej formie,
	\item omówienie sposobu formatowania zgodnego z~wymienionymi regułami,
	\item zamieszczenie przykładów użycia typowych obiektów urozmaicających treść z~wykorzystaniem natywnego środowiska edycji dokumentu,
	\item zaanonsowanie dobrych praktyk tworzenia krótkich form tekstowych.
\end{itemize}

Więcej informacji na temat przygotowania artykułu i~używania języka zainteresowany Czytelnik może znaleźć w~regulaminie przedmiotu i~poradni językowej PWN~\cite{ref:pwn}. W~celu zgłoszenia usterki merytorycznej sugeruje się kontakt z~prowadzącym, a~w~przypadku dostrzeżenia usterki technicznej -- należy zamieścić komentarz wraz z~propozycją zmiany w~repozytorium projektu \cite{ref:github}.
