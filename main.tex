\documentclass[runningheads]{class/class}

% Packages
\usepackage[
	a4paper,
	left=44mm,
	right=44mm,
	top=52mm,
	bottom=52mm
]{geometry}
\usepackage[polish]{babel}
\usepackage[utf8]{inputenc}
\usepackage[T1]{fontenc}
\usepackage[
	colorlinks=true,
	citecolor=black,
	linkcolor=black,
	urlcolor=blue,
	bookmarksopen=true
]{hyperref}
\usepackage{bm} % bold math symbols
\usepackage{tikz} % creating graphic vectors
\usepackage{pgfplots} % plots
\usepackage{enumitem} % control over itemize, enumerate
\usepackage{array}
\usepackage{algorithm}
\usepackage{algpseudocode} % texlive science
\usepackage{subfig}

% Paths
\graphicspath{{media/}}

% Geometry
\pgfplotsset{
	width=7.5cm,
	compat=1.12
}
\setlist{topsep=4pt}

% Commands
\renewcommand\UrlFont{\color{blue}\rmfamily} % blue roman url font, Springer's eBook style
\newcolumntype{A}{>{}m{2.2cm}}
\newcolumntype{B}{>{}m{9cm}}

\floatname{algorithm}{Listing} % name


\title{Zasady składania artykułu z wykorzystaniem zmodyfikowanego szablonu LNCS}
\titlerunning{Zasady składania artyk. z wykorzyst. zmodyfikowanego szablonu LNCS} % abbreviated if normal is too long
\author{Łukasz Marcul}
\authorrunning{Ł. Marcul}
\institute{Instytut Informatyki, Politechnika Warszawska\\
	\email{gall.anonim.stud@pw.edu.pl}
}

\begin{document}

\maketitle

\begin{abstract}
	W niniejszym dokumencie przedstawiono zestaw zasad składania artykułu dla potrzeb realizacji założeń Seminarium Dyplomowego 2 z~wykorzystaniem zaleceń formatowania wydawnictwa Springer. Omówiono reguły wykorzystania nagłówków i~popularnych konstrukcji tekstowych (tabele, twierdzenia, propozycje, itd.). Zasygnalizowano przykładowe błędy i~sposoby ich uniknięcia.

	\keywords{
		zalecenia formatowania artykułu,
		szablon LNCS.
	}
\end{abstract}

% Content
\section{Wprowadzenie}
\label{sec:intro}

Szablon Lecture Notes in Computer Science (LNCS) można wykorzystać do składania i edycji krótkich form tekstowych \cite{ref:lncs}. Dla potrzeb realizacji założeń Seminarium Dyplomowego, przygotowano bliźniacze, zmodyfikowane wersje z~wykorzystaniem języka \LaTeX~\cite{ref:latex} i~dokumentu Microsoft Word (z~makrami), które powinny pozytywnie wpłynąć na komfort pracy Studenta w realizacji zadania.

\subsection{Istota zadania}
\label{subsec:essence}

Jednym z wymagań przedmiotu Seminarium Dyplomowe jest opracowanie artykułu naukowego zawierającego treść referatu dotyczącego Pracy Dyplomowej Studenta. Dopuszcza się przygotowanie tekstu na temat informatyki, niezwiązany z~pracą dyplomową. W tym przypadku należy dowieść istotnego wkładu Studenta (np. poprzez analizę porównawczą metod lub autorskie eksperymenty).

Istotą zadania jest opracowanie nowego artykułu, dotyczącego nieprezentowanych dotychczas zagadnień. Na ocenę dokumentu wpływa jakość opracowania, w tym:

\begin{itemize}[noitemsep]
	\item spełnienie wymagań formalnych,
	\item stosowalność proponowanych rozwiązań,
	\item sposób użycia języka naukowo-technicznego,
	\item brak lub obecność błędów ortograficznych/literowych,
	\item poprawność formatowania tekstu.
\end{itemize}

\begin{note}
	Artykuł z rażącymi błędami redakcyjnymi lub językowymi, znamionującymi brak elementarnej staranności autora, \textbf{może być oceniony negatywnie bez pełnej weryfikacji tekstu}.
\end{note}

\begin{note}
	Treści w prezentacji, w szczególności ilustracje i~schematy blokowe, powinny być oryginalne lub umieszczone w~sposób nienaruszający prawa cytatu.
\end{note}

\subsection{Przygotowanie artykułu}
\label{subsec:preparation}

Student jest zobowiązany do przygotowania i~dostarczenia dokumentu PDF, sformatowanego zgodnie z~szablonem, umożliwiającego nanoszenie komentarzy, stosując zasady opisane poniżej.

\begin{table}
	\vspace{-4mm}
	\caption{
		Zasady przygotowania artykułu.
	}
	\begin{center}
		\begin{tabular}{AB}
			\hline
			Wymagania formalne & 6-8 stron tekstu przy nominalnym formatowaniu. Format papieru A4, marginesy pionowe i poziome, krój i~wielkość czcionek, interlinia, odstępy i~wcięcia akapitów, numeracja sekcji i~stron, wyrównanie, dzielenie wyrazów zdefiniowane przez szablon.\\
			\hline

			Tytuł & Możliwie krótki, zrozumiały i~adekwatny do zawartości artykułu.\\
			\hline

			Streszczenie & Powinno określać cel, zakres i~strukturę artykułu, a~co najmniej anonsować jego treść. Powinno zawierać słowa kluczowe, charakteryzujące przedmiot artykułu.\\
			\hline

			Treść & Zorganizowana w sekcje o czytelnej strukturze logicznej, zawartość sekcji adekwatna do ich tytułów. Powinna wskazywać źródła cytowanych stwierdzeń, hipotez, prognoz, ilustracji, tabel.\\
			\hline

			Podsumowanie & Adekwatne do celu i zakresu artykułu. Powinno zawierać wnioski, a co najmniej syntetyczną rekapitulację treści.\\
			\hline

			Bibliografia & Powinna wskazywać cytowane źródła w~sposób umożliwiający czytelnikowi dotarcie do ich treści. W przypadku źródeł internetowych powinna wskazywać datę odsłony.\\
			\hline
		\end{tabular}
	\end{center}
	\label{tab:rules}
	\vspace{-6mm}
\end{table}

\noindent Należy zadbać o czytelność:

\begin{itemize}[nosep]
	\item wskazywać rozwinięcia używanych skrótów, wyjaśniać specjalistyczne pojęcia przy pierwszym użyciu,
	\item unikać zbyt długich zdań, wtrąceń, przypisów dolnych i innych konstrukcji utrudniających czytanie,
	\item unikać określeń potocznych i żargonowych,
	\item zwrócić uwagę na łamanie linii i stron.
\end{itemize}

\section{Formatowanie}
\label{sec:formatting}

Aby uniknąć nieporozumień dotyczących formatowania poniżej podsumowano najważniejsze informacje jego dotyczące. Czcionki tytułu, i~trzech poziomów zagnieżdżeń sekcji powinny odpowiednio przyjąć rozmiar 14, 12, 10, 10 punktów i~być pogrubione.

\subsection{Podział tekstu}
\label{subsec:textDivision}

W niniejszym przypadku, pierwszy akapit sekcji (niezależnie od poziomu ich zagnieżdżenia) występuje z pominięciem wcięcia.

Następujące po nim akapity powinny być wcięte.

\subsubsection{Poziomy sekcji}
\label{subsubsec:levels}

Zauważmy, że wyłącznie dwa poziomy sekcji są numerowane. Dalsze zagnieżdżenie skutkuje brakiem numeracji - wymagany sposób formatowania tekstu w~tym przypadku to wlewanie nagłówków w~tej samej linii, tak jak opisano w tej sekcji.

\paragraph{Granice zagnieżdżania}
\label{par:nestingLimits}

Pomimo, że szablon LNCS Springer dopuszcza stosowanie czterech poziomów zagnieżdżenia, ostatni poziom najczęściej wpływa negatywnie na czytelność tekstu. Zainteresowany Czytelnik powinien stosować maksymalnie trzy opisywane poziomy, z~pominięciem poziomu zagnieżdżenia tej sekcji. Należy dążyć do rekomendowanej sytuacji, w~której stosowane są dwa poziomy zagnieżdżenia.

\subsubsection{Łamanie linii}
\label{subsubsec:linebreak}

Tekst powinien być wyjustowany. Część wyrazów w~języku polskim (np. krótkie spójniki) nie może być stosowana na końcu linii. Aby tego uniknąć, należy zastosować skrypt porządkujący tekst lub korzystać ze znaku twardej spacji.

\subsubsection{Łamanie stron}
\label{subsubsec:pagebreak}

Dobrą praktyką, nie zawsze możliwą do realizacji, jest przejście do nowej strony na granicach akapitów i~w~taki sposób, aby w treści akapitu kończącego stronę nie było odwołania do ilustracji/tabeli znajdującego się na kolejnej. Należy również unikać kontynuacji niewielkiej części wątku na następnej stronie, szczególnie jeśli jest parzysta (czyli niewidoczna w~trybie czytania dwóch stron jednocześnie).

\subsubsection{Odstępy pionowe}
\label{subsubsec:verticalSpace}

Nadmierne odstępy pionowe utrudniają odbiór tekstu. Próba wdrożenia zasady łamania stron może wpłynąć na zwiększenie tych odstępów - należy unikać takiej sytuacji. Czytelnik jest zobowiązany do przeredagowania tekstu aby była możliwość umieszczenia jeszcze jednego akapitu na stronie, gdy doświadczy opisanego problemu.

\subsection{Uzupełnienia treści}
\label{subsec:additions}

Wśród często stosowanych obiektów urozmaicających treść są tabele i~rysunki. Rozpatrzmy oba przykłady (Tab. \ref{tab:styles}, Rys. \ref{fig:devop}). Zauważmy, że opis powinien znajdować się powyżej tabeli, ale poniżej rysunku. Należy zwrócić uwagę na dopasowanie rozmiaru obiektów do pozostałej treści strony.

\begin{table}
	\vspace{-4mm}
	\caption{
		Style wlewania tekstu.
	}
	\begin{center}
		\begin{tabular}{lll}
			\hline
			Typ & Przykład & Styl i~wielkość czcionki\\
			\hline
			Tytuł & {\Large\bfseries Instrukcja} & 14 punktów, pogrubiona\\
			Sekcja &  {\large\bfseries 1 Wprowadzenie} & 12 punktów, pogrubiona\\
			Podsekcja & {\bfseries 2.1 Podział tekstu} & 10 punktów, pogrubiona\\
			Paragraf & {\bfseries Poziomy sekcji} & 10 punktów, pogrubiona\\
			Zwykły tekst & Nadmierne odstępy pionowe & 10 punktów\\
			\hline
		\end{tabular}
	\end{center}
	\label{tab:styles}
	\vspace{-6mm}
\end{table}

\begin{figure}
	\begin{center}
		\vspace{-5mm}
		\includegraphics[width=8cm]{device-operation.png}
		\caption{
			Diagram stanów pracy beacona (model uproszczony). Po uruchomieniu i~konfiguracji urządzenie przechodzi w tryb zmniejszonego poboru energii. Operacje rozgłaszania i odczytu/pomiaru są najczęściej wykonywane okresowo.
		}
		\label{fig:devop}
		\vspace{-8mm}
	\end{center}
\end{figure}

\noindent Uzupełnienia treści powinny nawiązywać co najmniej do jednego akapitu sekcji. W~ten sposób Czytelnik jest w stanie poznać intencje autora dotyczące prezentacji obiektów urozmaicających w~artykule. Zbyt krótkie nawiązania mogą być przejawem niedokładności autora tekstu, a zbyt długie - utrudniać zrozumienie jego rozważań.

\subsubsection{Wykresy}
\label{subsubsec:charts}

Korzystając z~tej formy prezentacji informacji, warto pamiętać o~opisaniu osi liczbowych, użyciu adekwatnej skali i~typie wykresu dostosowanym do rodzaju danych, a~także konsekwencji stylistycznej.

\begin{figure}[!h]
	\vspace{-3mm}
	\centering
	\begin{tikzpicture}
		\begin{axis}[
			xlabel={zestaw danych},
			ylabel={rozmiar [b]},
			width=0.9\textwidth,
			height=5.5cm,
			xmin = 1,
			xmax = 90,
			ymax = 85000,
			minor y tick num = 4,
			legend columns=-1
		]
			\addplot table[
				x=lp,y=prop,mark=none
			]{results.csv}; \addlegendentry{propozycja}
			\addplot table[
				x=lp,y=rle,mark=none
			]{results.csv}; \addlegendentry{k. długości serii}
			\addplot table[
				x=lp,y=huffman,mark=none
			]{results.csv}; \addlegendentry{k. Huffmana}
			\draw [dashed] (0, 46080) -- (90, 46080);
		\end{axis}
	\end{tikzpicture}
	\caption{
		Uzyskany rozmiar danych po skorzystaniu z omawianych algorytmów dla zestawów przyrostów wartości liczydła energii czynnej. Linią przerywaną zaznaczono rozmiar przed kompresją. Próba: 90 instalacji na różnych licznikach energii, zbiór danych z~pełnego miesiąca.
	}
	\vspace{-6mm}
	\label{fig:compressionAlgorithms}
\end{figure}

\subsubsection{Wnioskowanie logiczne}
\label{subsubsec:logic}

Pytania, spostrzeżenia, teorie i~hipotezy, lematy, twierdzenia, dowody i wnioski, definicje, przykłady, własności i~rozwiązania - zbiór metodologii zawiera wiele form prezentacji logicznych wyrażeń. Student może je (oszczędnie) stosować, aby uporządkować tok rozumowania.

\begin{claim}
	Każdy wzór matematyczny skraca o~połowę liczbę aktywnych użytkowników prezentacji.
\end{claim}
\begin{proof}
	Załóżmy, że istnieje wzór, który nie skraca liczby aktywnych użytkowników prezentacji o połowę. Przyjmijmy dla ustalenia uwagi, że jest postaci jak poniżej.

	\begin{equation} \label{eq:formula}
	e^{\pi i} + 1 = 0
	\end{equation}

	\noindent Tożsamość Eulera (\ref{eq:formula}), jest uznawana za najpiękniejszy wzór matematyczny z~uwagi na wystąpienie trzech działań arytmetycznych łączących pięć fundamentalnych stałych matematycznych. Ponadto, prezentowane równanie jest wyśrodkowane w~nowej linii (zgodnie ze sztuką). Większość czytelników nadal powinna być skupiona na treści, jednak intuicja podpowiada, że zastanawiają się oni nad sensownością argumentacji, co prowadzi do niespodziewanej sprzeczności. \qed
\end{proof}

Wyciąganie niedokładnych wniosków stanowi o poziomie ignorancji autora. Sugeruje się krytyczne podejście do własnych badań i~intuicji, precyzyjny opis procedury i~środowiska testowego, a~także wnikliwą analizę otrzymanych rezultatów.

\subsubsection{Pseudokod}
\label{subsubsec:pseudocode}

Zaleca się stosowanie pseudokodu aby przedstawić zasadę działania algorytmu. Czytelnik powinien stosować minimalistyczne podejście w~opisie obiektów - jeśli wyłącznie jeden obiekt danego typu występuje, dobrą praktyką jest ukrycie jego numeru, jak na przykładzie poniżej.

\vspace{-4mm}
\begin{algorithm}
	\renewcommand{\thealgorithm}{} % comment it if multiple algorithms exist
	\caption{Algorytm Euklidesa} \label{alg:euclid}
	\begin{algorithmic}[1]
		\Procedure{NWD}{$a,b$} \Comment{Największy wspólny dzielnik liczb a i b}
			\State $r\gets a\bmod b$
			\While{$r\not=0$} \Comment{W przeciwnym przypadku odpowiedź jest trywialna}
				\State $a\gets b$
				\State $b\gets r$
				\State $r\gets a\bmod b$
			\EndWhile \label{alg:euclid:endwhile}
			\State \textbf{return} $b$
		\EndProcedure
	\end{algorithmic}
\end{algorithm}
\vspace{-8mm}

\subsubsection{Cytowanie}
\label{subsubsec:cite}

Preferowanym stylem cytowania są liczby w nawiasach kwadratowych, uporządkowane według kolejności wystąpienia w~artykule \cite{ref:lncs}. Dopuszcza się stosowanie etykiet lub nazwisk, jednak te podejście może wpłynąć negatywnie na czytelność tekstu. Cytowania wielu źródeł powinny być zgrupowane \cite{ref:lncs,ref:latex}, a~wszystkie pozycje literatury - mieć swoje odwołania w tekście.

\subsubsection{Inne konstrukcje}
\label{subsubsec:others}

Zachęca się do niestosowania innych konstrukcji uzupełniających treść, ponieważ mogą one wpłynąć negatywnie na odbiór artykułu przez Czytelnika. Nadużywanie ich w~większości przypadków prowadzi do niezrozumienia treści lub trywializacji zawartości dokumentu.

\section{Podsumowanie}
\label{sec:summary}

Jednym z~elementów pracy twórczej Studenta realizującego przedmiot Seminarium Dyplomowe jest dostarczenie nowego artykułu, dotyczącego nieprezentowanych dotychczas zagadnień, przygotowanego zgodnie z szablonem, w formacie PDF, umożliwiającego nanoszenie komentarzy. Autor powinien odwołać się do wszystkich pozycji literatury z~poszanowaniem prawa cytatu. Na ocenę wpływa jakość opracowania i~czytelność. Dokument z~rażącymi błędami redakcyjnymi lub językowymi może być oceniony negatywnie bez pełnej weryfikacji tekstu.

Pierwszy akapit sekcji powinien wystąpić z~pominięciem wcięcia, kolejne - powinny być wcięte. Preferuje się korzystanie z~dwóch lub trzech poziomów zagnieżdżeń. Należy zwrócić uwagę na poprawne łamanie linii i~stron, między innymi przejście do nowej strony na granicy akapitów i~unikanie odwołań do ilustracji/tabel na kolejnych stronach.

Zalecanymi uzupełnieniami treści są tabele, rysunki, schematy, wykresy, wnioskowania logiczne, pseudokod i~cytowania. Rozmiar wspomnianych obiektów powinien być adekwatny do otaczającej treści, a~użycie uzasadnione. Zbyt krótkie nawiązania w~treści do nich mogą być przejawem niedokładności autora, a~zbyt długie - utrudniać zrozumienie jego rozważań.

Należy oszczędnie i~precyzyjnie dowodzić i~wnioskować, aby uporządkować tok rozumowania. Artykuł powinien mieć wyraźną strukturę logiczną, zwykle w~postaci rozumowania dedukcyjnego (\emph{od ogółu do szczegółu}): po krótkim wprowadzeniu prezentowane są cechy konkretnego rozwiązania z~wykorzystaniem adekwatnej metodologii (np. porównanie rozwiązań przy użyciu kryterium ilościowego), następuje detaliczna analiza tematu, po czym Czytelnik zobowiązany jest do syntetycznej rekapitulacji treści.

Więcej informacji na temat dobrych praktyk tworzenia artykułu i~używania języka zainteresowany Czytelnik może znaleźć w~regulaminie przedmiotu i~poradni językowej PWN~\cite{ref:pwn}. Niniejszy materiał przeznaczony jest dla Studentów realizujących przedmiot Seminarium Dyplomowe w~Instytucie Informatyki Politechniki Warszawskiej. W~przypadku chęci zgłoszenia usterki w~pierwszej kolejności sugeruje się kontakt z~prowadzącym, następnie zamieszczenie komentarza i~propozycji zmiany w~repozytorium projektu \cite{ref:github}.


\begin{thebibliography}{}
\bibitem{ref:llncs}
Springer: Information for Authors of Springer Computer Science Proceedings. Dostęp online (20.12.2019):
\url{https://www.springer.com/gp/computer-science/lncs/conference-proceedings-guidelines}.

\end{thebibliography}

\end{document}
