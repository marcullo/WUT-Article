\documentclass[runningheads]{class/class}

% Packages
\usepackage[
	a4paper,
	left=44mm,
	right=44mm,
	top=52mm,
	bottom=52mm
]{geometry}
\usepackage[polish]{babel}
\usepackage[utf8]{inputenc}
\usepackage[T1]{fontenc}
\usepackage[
	colorlinks=true,
	citecolor=black,
	linkcolor=black,
	urlcolor=blue,
	bookmarksopen=true
]{hyperref} % hyperref is necessary for \emailsubj of used class
\usepackage{bm} % bold math symbols
\usepackage{tikz} % creating graphic vectors
\usepackage{pgfplots} % plots
\usepackage{enumitem} % control over itemize, enumerate
\usepackage{array}
\usepackage{algorithm}
\usepackage{algpseudocode} % texlive science
\usepackage{subfig}
\usepackage{caption}

% Paths
\graphicspath{{media/}}

% Geometry
\pgfplotsset{
	width=7.5cm,
	compat=1.12
}
\setlist{topsep=4pt}

% Caption font
% https://en.wikibooks.org/wiki/LaTeX/Fonts#Built-in_sizes
\captionsetup[figure]{font=small, labelfont=bf}
\captionsetup[table]{font=small, labelfont=bf}

% Commands
\renewcommand\UrlFont{\color{blue}\rmfamily} % blue roman url font, Springer's eBook style
\newcolumntype{A}{>{}m{2.2cm}}
\newcolumntype{B}{>{}m{9cm}}
\newcommand\dummyemail{
	\hypersetup{urlcolor=black}
	\emailsubj[gall.anonim.stud@pw.edu.pl]{Warning! Provided e-mail address is dummy. Do not send a message.}
}

\floatname{algorithm}{Listing} % name


\title{Zasady składania artykułu z wykorzystaniem zmodyfikowanego szablonu LNCS}
\titlerunning{Zasady składania artyk. z wykorzyst. zmodyfikowanego szablonu LNCS.} % abbreviated if normal is too long
\author{Gall Anonim}
\authorrunning{G. Anonim}
\institute{Instytut Informatyki, Politechnika Warszawska\\
	\dummyemail % remove and use valid email instead
%	\email{gall.anonim.stud@pw.edu.pl}
}

\begin{document}

\maketitle

\begin{abstract}
	W niniejszym dokumencie o~wersji \input{../VERSION}przedstawiono zestaw zasad składania artykułu dla potrzeb realizacji założeń Seminarium Dyplomowego Magisterskiego 2 z~wykorzystaniem zaleceń formatowania wydawnictwa Springer. Omówiono reguły wykorzystania nagłówków i~popularnych konstrukcji tekstowych (tabele, twierdzenia, propozycje, itd.). Zasygnalizowano przykładowe błędy i~sposoby ich uniknięcia.

	\keywords{
		zalecenia formatowania artykułu,
		szablon LNCS.
	}
\end{abstract}

% Content
\section{Wprowadzenie}
\label{sec:intro}

Szablon Latex Lecture Notes in Computer Science (LLNCS) można wykorzystać do składania i edycji krótkich form tekstowych \cite{ref:llncs}. Dla potrzeb realizacji założeń Seminarium Dyplomowego, przygotowano bliźniacze, zmodyfikowane, wersje z~wykorzystaniem języka \LaTeX~i~dokumentu Microsoft Word (z~makrami), które powinny pozytywnie wpłynąć na komfort pracy Studenta w realizacji zadania.

\subsection{Istota zadania}
\label{subsec:essence}

Jednym z wymagań przedmiotu Seminarium Dyplomowe jest opracowanie artykułu naukowego zawierającego treść referatu dotyczącego Pracy Dyplomowej Studenta. Dopuszcza się przygotowanie tekstu na temat informatyki, niezwiązany z~pracą dyplomową. W tym przypadku należy dowieść istotnego wkładu Studenta (np. poprzez analizę porównawczą metod lub autorskie eksperymenty).

Istotą zadania jest opracowanie nowego artykułu, dotyczącego nieprezentowanych dotychczas zagadnień. Na ocenę dokumentu wpływa jakość opracowania, w tym:

\begin{itemize}[noitemsep]
	\item spełnienie wymagań formalnych,
	\item stosowalność proponowanych rozwiązań,
	\item sposób użycia języka naukowo-technicznego,
	\item brak lub obecność błędów ortograficznych/literowych,
	\item poprawność formatowania tekstu.
\end{itemize}

\begin{note}
	Artykuł z rażącymi błędami redakcyjnymi lub językowymi, znamionującymi brak elementarnej staranności autora, \textbf{może być oceniony negatywnie bez pełnej weryfikacji tekstu}.
\end{note}

\begin{note}
	Treści w prezentacji, w szczególności ilustracje i~schematy blokowe, powinny być oryginalne lub umieszczone w~sposób nienaruszający prawa cytatu.
\end{note}

\subsection{Przygotowanie artykułu}
\label{subsec:preparation}

Student jest zobowiązany do przygotowania i~dostarczenia dokumentu PDF, sformatowanego zgodnie z~szablonem, umożliwiającego nanoszenie komentarzy, stosując zasady opisane poniżej.

\begin{table}
	\vspace{-4mm}
	\caption{
		Zasady przygotowania artykułu.
	}
	\begin{center}
		\begin{tabular}{AB}
			\hline
			Wymagania formalne & 6-8 stron tekstu przy nominalnym formatowaniu. Format papieru A4, marginesy pionowe i poziome, krój i~wielkość czcionek, interlinia, odstępy i~wcięcia akapitów, numeracja sekcji i~stron, wyrównanie, dzielenie wyrazów zdefiniowane przez szablon.\\
			\hline

			Tytuł & Możliwie krótki, zrozumiały i~adekwatny do zawartości artykułu.\\
			\hline

			Streszczenie & Powinno określać cel, zakres i~strukturę artykułu, a~co najmniej anonsować jego treść. Powinno zawierać słowa kluczowe, charakteryzujące przedmiot artykułu.\\
			\hline

			Treść & Zorganizowana w sekcje o czytelnej strukturze logicznej, zawartość sekcji adekwatna do ich tytułów. Powinna wskazywać źródła cytowanych stwierdzeń, hipotez, prognoz, ilustracji, tabel.\\
			\hline

			Podsumowanie & Adekwatne do celu i zakresu artykułu. Powinno zawierać wnioski, a co najmniej syntetyczną rekapitulację treści.\\
			\hline

			Bibliografia & Powinna wskazywać cytowane źródła w~sposób umożliwiający czytelnikowi dotarcie do ich treści. W przypadku źródeł internetowych powinna wskazywać datę odsłony.\\
			\hline
		\end{tabular}
	\end{center}
	\label{tab:rules}
	\vspace{-6mm}
\end{table}

\noindent Należy zadbać o czytelność:

\begin{itemize}[nosep]
	\item wskazywać rozwinięcia używanych skrótów, wyjaśniać specjalistyczne pojęcia przy pierwszym użyciu,
	\item unikać zbyt długich zdań, wtrąceń, przypisów dolnych i innych konstrukcji utrudniających czytanie,
	\item unikać określeń potocznych i żargonowych,
	\item zwrócić uwagę na łamanie linii i stron.
\end{itemize}

\section{Formatowanie}
\label{sec:formatting}

\subsection{Model referencyjny}
\label{subsec:refmodel}

W niniejszym przypadku, pierwszy akapit sekcji (niezależnie od poziomu ich zagnieżdżenia) występuje z pominięciem wcięcia.

Następujące po nim akapity powinny być wcięte.

\subsubsection{Poziomy sekcji}
\label{subsubsec:levels}

Zauważmy, że wyłącznie dwa poziomy sekcji są numerowane. Dalsze zagnieżdżenie skutkuje brakiem numeracji - wymagany sposób formatowania to nagłówki w~tej samej linii, tak jak w~tym przypadku.

\paragraph{title}
\label{}

Pomimo, że szablon LLNCS Springer dopuszcza stosowanie czterech poziomów zagnieżdżenia, ostatni poziom najczęściej wpływa negatywnie na czytelność tekstu.
\section{Podsumowanie}
\label{sec:summary}

Najmniej wartościowym, dopuszczalnym sposobem podsumowania jest syntetyczna rekapitulacja treści. Wartością dodaną, świadczącą o~jakości puenty jest interpretacja uzyskanych wyników, w~szczególności ich odniesienie do rezultatów innych prac i~wyjaśnienie ewentualnych różnic. Podsumowanie jest właściwym miejscem do przedstawienia uzyskanych rezultatów w~szerszym kontekście (np.~biznesowym lub społecznym) i~wskazania kierunków dalszych prac. Poniżej podsumowano (pokrótce) artykuł z~uwzględnieniem powyższej treści.

Jednym z~elementów pracy twórczej Studenta realizującego Seminarium Dyplomowe~2 jest dostarczenie nowego artykułu, dotyczącego nieprezentowanych dotychczas (na seminarium) zagadnień, przygotowanego zgodnie z~szablonem. Na ocenę wpływa jakość opracowania i~czytelność. Należy zwrócić uwagę na poprawność formatowania, między innymi właściwe łamanie linii i~stron oraz unikanie odwołań do ilustracji/tabel na kolejnych stronach. Artykuł powinien mieć wyraźną strukturę logiczną, zakończoną syntetyczną informacją o~zakresie zrealizowanych prac i~wartością dodaną, opisaną w~poprzednim akapicie.

Dokument opracowano w~odpowiedzi na zapotrzebowanie na jednolity szablon artykułu w~języku \LaTeX~i~programie Microsoft Word zgłoszone przez grupę studentów realizujących przedmiot, którego niniejszy tekst dotyczy. Do cech wyróżniających artykułu można zaliczyć:

\begin{itemize}
	\item zawarcie istoty zadania i~zasad przygotowania artykułu w~logicznej formie,
	\item omówienie sposobu formatowania zgodnego z~wymienionymi regułami,
	\item zamieszczenie przykładów użycia typowych obiektów urozmaicających treść z~wykorzystaniem natywnego środowiska edycji dokumentu,
	\item zaanonsowanie dobrych praktyk tworzenia krótkich form tekstowych.
\end{itemize}

Więcej informacji na temat przygotowania artykułu i~używania języka zainteresowany Czytelnik może znaleźć w~regulaminie przedmiotu i~poradni językowej PWN~\cite{ref:pwn}. W~przypadku chęci zgłoszenia usterki w~pierwszej kolejności sugeruje się kontakt z~prowadzącym, następnie zamieszczenie komentarza wraz z~propozycją zmiany w~repozytorium projektu \cite{ref:github}.

\bibitem{ref:llncs}
Springer: Information for Authors of Springer Computer Science Proceedings. Dostęp online (20.12.2019):
\url{https://www.springer.com/gp/computer-science/lncs/conference-proceedings-guidelines}.


\end{document}
